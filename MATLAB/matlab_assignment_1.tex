\documentclass{article}
\usepackage{graphicx}
\usepackage{gensymb}
\usepackage{amsmath}
\usepackage{listings}
\usepackage{amssymb}
\title{\textbf{MATLAB ASSIGNMENT 1}}
\author{Pritha Jana,IMS21276}
\date{}
\begin{document}
\maketitle
\begin{enumerate}
\item MATLAB code to solve a nonlinear equation $f(x)= 0$ using the following methods and test it for $f(x) = sin(x)+x^2-1$ take the interval $[0,1]$.
\begin{enumerate}
\item \textbf{Bisection Method}\\
\underline{Code}:
\begin{lstlisting}
%Bisection method 
close all;
close all;

upper_bound="Let's choose an upper bound--(a1)--"
a1=input(upper_bound)
lower_bound="Let's choose an lower bound--(b1)--"
b1=input(lower_bound)
fun1=@(x) sin(x)+x^2-1;
c=(a1+b1)/2;
i=0;
data=[i a1 b1 c fun1(c)];
while (abs(fun1(c)) > 10^(-4))
if fun1(a1)*fun1(c)<0
    b1=c;
else
    a1=c;
end
i=i+1;
c=(a1+b1)/2;
data=[i a1 b1 c fun1(c)]

end

\end{lstlisting}

\underline{Result}: data=[ 7.0000    0.6406    0.6328    0.6367   -0.0000]
\item \textbf{ Newton-Raphson Method}\\
\underline{Code}:
\begin{lstlisting}
%Newton-Raphson Method
close all;
close all;

nearest_point="Let's choose a nearest point--(x0)--"
x0=input(nearest_point)
fun1 =@(x) sin(x)+x^2-1.0;
fun1(x0)
syms f(x)
f(x) = sin(x)+x^2;
Df = diff(f,x);
D=double(Df(x0))
i=0;
x=x0-(fun1(x0)/D)
abs(fun1(x))
while (abs(fun1(x))>10^(-8))
i=i+1;
x=x-(fun1(x)/D);
data=[i x fun1(x)]
end
\end{lstlisting}
\underline{Result}: data=[10.0000    0.6367    0.0000]
\item \textbf{Secant Methods}\\
\underline{Code}:
\begin{lstlisting}
%Secant Method 
close all;
close all;
a1="Let's choose a nearest point--(x0)--"
x0=input(a1)
a2="Let's choose a nearest point--(x1)--"
x1=input(a2)
fun1 =@(x) sin(x)+x^2-1.0;
i=0;
x=x1-(((x0-x1)/(fun1(x0)-fun1(x1)))*fun1(x1));
while (abs(fun1(x)) > 10^(-4))
if fun1(x0)*fun1(x1)<0
    x0=x;
else
    x1=x;
end
i=i+1;
x=x1-((x0-x1)/(fun1(x0)-fun1(x1)))*fun1(x1)
data=[i x0 x1 x fun1(x)]

end           

\end{lstlisting}
\underline{Result}: data=[4.0000    0.6366    1.0000    0.6367   -0.0000]

\item \textbf{Regula-Falsi Method}\\
\underline{Code}:
\begin{lstlisting}
%Regula-falsi method
close all;
close all;
upper_bound="Let's choose an upper bound--(a1)--"
a1=input(upper_bound)
lower_bound="Let's choose an lower bound--(b1)--"
b1=input(lower_bound)
fun1=@(x) sin(x)+x^2-1;
x=a1-(fun1(a1)*((b1-a1)/(fun1(b1)-fun1(a1))))
i=0;
data=[i a1 b1 x fun1(x)];
while (abs(fun1(x)) > 10^(-4))
if fun1(a1)*fun1(x)<0
    b1=x;
else
    a1=x;
end
i=i+1;
x=a1-(fun1(a1)*((b1-a1)/(fun1(b1)-fun1(a1))))
data=[i a1 b1 x fun1(x)]

end
\end{lstlisting}
\underline{Result}: data=[4.0000    1.0000    0.6366    0.6367   -0.0000]

\item \textbf{Fixed-point Iteration Method}\\
\underline{Code}:
\begin{lstlisting}
%Fixed-point Iteration Method
close all;
close all;
upper_bound="Let's choose any number--(x0)--"
x0=input(upper_bound)
fun1=@(x) sin(x)+x^2-1.0;
g=@(x) sqrt(1-sin(x));
i=0;
fun1(x0)
x1=g(x0);
data=[i x1 fun1(x1)]
while (abs(fun1(x0))>10^(-4))
x0=g(x0)
x1=g(x0)
i=i+1;
data=[i  x1   fun1(x1)]
end
   
\end{lstlisting}
\underline{Result}: data=[20.0000    0.6367   -0.0000]
\end{enumerate}
\item Apply Newton-Raphson method to approximate the root of equation $f(x)=x^3-x-3=0$ with initial guess $x0=0$.Show that the sequence diverges. Further,perform the Newton the Newton-Raphson method with initial guess sufficiently close to the root $r\approx1.6717$ and discuss the convergence in this case.
\begin{enumerate}
\item By Choosing nearest point $x0=0$\\
\underline{Code}:
\begin{lstlisting}
%Newton-Raphson Method
close all;
close all;

nearest_point="Let's choose a nearest point--(x0)--"
x0=input(nearest_point)
fun1 =@(x) x^3-x-3;
fun1(x0)
syms f(x)
f(x) = sin(x)+x^2;
Df = diff(f,x);
D=double(Df(x0))
i=0;
x=x0-(fun1(x0)/D)
abs(fun1(x))
while (abs(fun1(x))>10^(-8))
i=i+1;
x=x-(fun1(x)/D);
data=[i x fun1(x)]
end
\end{lstlisting}
\underline{Result}: data=[7  -Inf   NaN]
\item By choosing the point $x0=2$ as the root is given $r\approx 1.6717$
\underline{Code}
\begin{lstlisting}
%Newton-Raphson Method
close all;
close all;

nearest_point="Let's choose a nearest point--(x0)--"
x0=input(nearest_point)
fun1 =@(x) x^3-x-3;
fun1(x0)
syms f(x)
f(x) = sin(x)+x^2;
Df = diff(f,x);
D=double(Df(x0))
i=0;
x=x0-(fun1(x0)/D)
abs(fun1(x))
while (abs(fun1(x))>10^(-8))
i=i+1;
x=x-(fun1(x)/D);
data=[i x fun1(x)]
end  
\end{lstlisting}
\underline{Result}: data=[ 1.0e+05*1.9751    0.0000    0.0000]\\
The sequence $f(x)=x^3-x-3$ itself is a diverging sequence.
\end{enumerate}
\item Consider the equation $x^2-6x+5=0$.
\begin{enumerate}
\item Taking $x0=0$ and $x1=4.5$, generate first 7 terms of the iterative sequence of the secant method.
\item Take the initial interval as $[a0, b0]=[0, 4.5]$, generate the first 7 terms of the iterative sequence of the regula-falsi method.
Observe to which roots of the given equation does the above two sequences converge?
\end{enumerate}
\begin{enumerate}
    \item Using Sccant Method \\
    \underline{Code}:
    \begin{lstlisting}
%Secant Method 
close all;
close all;
a1="Let's choose a nearest point--(x0)--"
x0=input(a1)
a2="Let's choose a nearest point--(x1)--"
x1=input(a2)
fun1 =@(x) x^2-6*x+5;
i=0;
x=x1-(((x0-x1)/(fun1(x0)-fun1(x1)))*fun1(x1));
while (abs(fun1(x)) > 10^(-4))
if fun1(x0)*fun1(x1)<0
    x0=x;
else
    x1=x;
end
i=i+1;
x=x1-((x0-x1)/(fun1(x0)-fun1(x1)))*fun1(x1)
data=[i x0 x1 x fun1(x)]

end  
    \end{lstlisting}
\underline{Result}:data=[ 7.0000    5.0000    5.4545    5.0000   -0.0000]\\
Root converges to 5.0000.
\item Using Regula-Falsi Method\\
\underline{Code}:
\begin{lstlisting}
%Regula-falsi method
close all;
close all;
upper_bound="Let's choose an upper bound--(a1)--"
a1=input(upper_bound)
lower_bound="Let's choose an lower bound--(b1)--"
b1=input(lower_bound)
fun1 =@(x) x^2-6*x+5;
x=a1-(fun1(a1)*((b1-a1)/(fun1(b1)-fun1(a1))))
i=0;
data=[i a1 b1 x fun1(x)];
while (abs(fun1(x)) > 10^(-4))
if fun1(a1)*fun1(x)<0
    b1=x;
else
    a1=x;
end
i=i+1;
x=a1-(fun1(a1)*((b1-a1)/(fun1(b1)-fun1(a1))))
data=[i a1 b1 x fun1(x)]

end

\end{lstlisting}
\underline{Result}: data=[ 7.0000    1.0004   0  1.0001  -0.0003]\\
Root converges to 1.0001 .
\end{enumerate}
\item Write a MATLAB code to solve a nonlinear equation f(x) = 0 using the Fixed-point iteration Method Consider the equation $f(x)=sin(x)+x^2-1$. Take the initial interval as [0,1]. There are three possible choices for the iteration functions namely\begin{enumerate} \item $g_1(x) =\sin^{-1}{(1-x^2)}$, \item $g_2(x) =-\sqrt{1-sin(x)}$, \item $g_3(x) =\sqrt{1-sin(x)}$.\end{enumerate}
Discuss the convergence or divergence of all the iterative sequences. Can you Justify theoretically?
\begin{enumerate}
    \item Using $g_1(x) =\sin^{-1}{(1-x^2)}$
    \begin{lstlisting}
% Fixed-point Iteration Method
close all;
close all;
upper_bound="Let's choose any number--(x0)--"
x0=input(upper_bound)
fun1=@(x) sin(x)+x^2-1.0;
g=@(x) asin(1-x^2)
i=0;
fun1(x0)
x1=g(x0);
data=[i x1 fun1(x1)]
while (abs(fun1(x0))>10^(-4))
x0=g(x0)
x1=x0
i=i+1;
data=[i  x1   fun1(x1)]
end  
    \end{lstlisting}
    \underline{Result}: data=[1.0e+04 *
4.0239 + 0.0000i   0.0001 + 0.0003i   0.0000 + 0.0013i
x0 =
Operation terminated by user during untitled2]\\
 Not convergent.


\underline{Theoretical proof}:

$g_1$ is not contraction map for $x\in[0,1]$.
\begin{align}
    g'(x)&=\frac{-2x}{\sqrt{1-(1-x^2)^2}}\\
    &=\frac{-2}{\sqrt{2-x^2}}\\
    \lambda &=\max_{\operatornamewithlimits{0 \le x \le 1}}|g'(x)|\nless1
\end{align}
    \item Using $g_2(x) =-\sqrt{1-sin(x)}$
    \begin{lstlisting}
% Fixed-point Iteration Method
close all;
close all;
upper_bound="Let's choose any number--(x0)--"
x0=input(upper_bound)
fun1=@(x) sin(x)+x^2-1.0;
g=@(x) -sqrt(1-sin(x))
i=0;
fun1(x0)
x1=g(x0);
data=[i x1 fun1(x1)]
while (abs(fun1(x0))>10^(-4))
x0=g(x0)
x1=x0
i=i+1;
data=[i  x1   fun1(x1)]
end
    \end{lstlisting}
  \underline{Result}: data=[  6.0000   -1.4096   -0.0000]\\
Not Convergent. \\
\underline{Theoretical proof}:\\
$g_2$ is not self map of $[0,1]$ to itself.
\begin{align}
     g_2(x) &=-\sqrt{1-sin(x)}\\
     g_2'(x)&=\frac{\sqrt{1+\sin{x}}}{2}\\
\end{align}

   \item Using $g_3(x) =\sqrt{1-sin(x)}$
    \begin{lstlisting}
% Fixed-point Iteration Method
close all;
close all;
upper_bound="Let's choose any number--(x0)--"
x0=input(upper_bound)
fun1=@(x) sin(x)+x^2-1.0;
g=@(x) sqrt(1-sin(x))
i=0;
fun1(x0)
x1=g(x0);
data=[i x1 fun1(x1)]
while (abs(fun1(x0))>10^(-4))
x0=g(x0)
x1=x0
i=i+1;
data=[i  x1   fun1(x1)]
end  
    \end{lstlisting}
 \underline{Result}: data=[ 20.0000    0.6368    0.0001]\\
 Convergent.\\
\underline{Theoretical proof}:\\
$g_3$ is converging to the root.
\begin{align}
    g_3(x)&=\sqrt{1-\sin{x}}\\
    g_3'(x)&=-\frac{\sqrt{1+\sin{x}}}{2}\\
    |g_3'(x)|&\le\frac{1}{\sqrt{2}}<1
\end{align}
\end{enumerate}
\end{enumerate}
\end{document}
